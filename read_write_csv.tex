\documentclass[a4paper]{article}

\usepackage[english]{babel}
\usepackage[utf8]{inputenc}
\usepackage{amsmath}

\usepackage{xcolor}

\definecolor{verde}{rgb}{0.25,0.5,0.35}
\definecolor{jpurple}{rgb}{0.5,0,0.35}
\definecolor{darkgreen}{rgb}{0.0, 0.2, 0.13}
%\definecolor{oldmauve}{rgb}{0.4, 0.19, 0.28}
\usepackage{listings}

\usepackage{tikz}
\def\checkmark{\tikz\fill[scale=0.4](0,.35) -- (.25,0) -- (1,.7) -- (.25,.15) -- cycle;} 


\newcommand{\cppTheme}{
	\lstset{
		language=Java,
		basicstyle=\ttfamily\small,
		keywordstyle=\color{jpurple}\bfseries,
		stringstyle=\color{red},
		commentstyle=\color{verde},
		morecomment=[s][\color{blue}]{/**}{*/},
		extendedchars=true,
		showspaces=false,
		showstringspaces=false,
		numbers=left,
		numberstyle=\tiny,
		breaklines=true,
		backgroundcolor=\color{cyan!10},
		breakautoindent=true,
		captionpos=b,
		xleftmargin=0pt,
		tabsize=2
}}

\newcommand{\test}{
	\lstset{ %
		language=R,                     % the language of the code
		basicstyle=\footnotesize,       % the size of the fonts that are used for the code
		numbers=left,                   % where to put the line-numbers
		numberstyle=\tiny\color{gray},  % the style that is used for the line-numbers
		stepnumber=1,                   % the step between two line-numbers. If it's 1, each line
		% will be numbered
		numbersep=5pt,                  % how far the line-numbers are from the code
		backgroundcolor=\color{white},  % choose the background color. You must add \usepackage{color}
		showspaces=false,               % show spaces adding particular underscores
		showstringspaces=false,         % underline spaces within strings
		showtabs=false,                 % show tabs within strings adding particular underscores
		frame=single,                   % adds a frame around the code
		rulecolor=\color{black},        % if not set, the frame-color may be changed on line-breaks within not-black text (e.g. commens (green here))
		tabsize=2,                      % sets default tabsize to 2 spaces
		captionpos=b,                   % sets the caption-position to bottom
		breaklines=true,                % sets automatic line breaking
		breakatwhitespace=false,        % sets if automatic breaks should only happen at whitespace
		title=\lstname,                 % show the filename of files included with \lstinputlisting;
		% also try caption instead of title
		keywordstyle=\color{blue},      % keyword style
		commentstyle=\color{darkgreen},   % comment style
		stringstyle=\color{red},      % string literal style
		escapeinside={\%*}{*)},         % if you want to add a comment within your code
		morekeywords={*,...}          % if you want to add more keywords to the set
}}

\title{Reading and Writing '.csv' files in C++}
\author{Johnjimy K. Som}

\begin{document}
	\maketitle
	
	\begin{scriptsize}%begin{abstract} (can be used here)
		Documentation of a C++ program 'read-write-csv.cpp' that reads and writes copies of '.csv' files using additional #includes. The program also creates additional copies that checks that the .csv file has been read. More information can be found in: https://www.gormanalysis.com/blog/reading-and-writing-csv-files-with-cpp/#writing-to-csv
	\end{scriptsize}%\end{abstract}
	
	\section{read-write-csv.cpp}
	\begin{scriptsize}
		\cppTheme
		\begin{lstlisting}[caption={read and write '.csv' in one '.cpp' file}, label=lst:rcode]
			#include <string>
			#include <fstream>
			#include <vector>
			#include <utility> // std::pair
			#include <stdexcept> // std::runtime_error
			#include <sstream> // std::stringstream
			
			std::vector<std::pair<std::string, std::vector<int>>> read_csv(std::string filename) {
				// Reads a CSV file into a vector of <string, vector<int>> pairs where
				// each pair represents <column name, column values>
				
				// Create a vector of <string, int vector> pairs to store the result
				std::vector<std::pair<std::string, std::vector<int>>> result;
				
				// Create an input filestream
				std::ifstream myFile(filename);
				
				// Make sure the file is open
				if (!myFile.is_open()) throw std::runtime_error("Could not open file");
				
				// Helper vars
				std::string line, colname;
				int val;
				
				// Read the column names
				if (myFile.good())
				{
					// Extract the first line in the file
					std::getline(myFile, line);
					
					// Create a stringstream from line
					std::stringstream ss(line);
					
					// Extract each column name
					while (std::getline(ss, colname, ',')) {
						
						// Initialize and add <colname, int vector> pairs to result
						result.push_back({ colname, std::vector<int> {} });
					}
				}
				
				// Read data, line by line
				while (std::getline(myFile, line))
				{
					// Create a stringstream of the current line
					std::stringstream ss(line);
					
					// Keep track of the current column index
					int colIdx = 0;
					
					// Extract each integer
					while (ss >> val) {
						
						// Add the current integer to the 'colIdx' column's values vector
						result.at(colIdx).second.push_back(val);
						
						// If the next token is a comma, ignore it and move on
						if (ss.peek() == ',') ss.ignore();
						
						// Increment the column index
						colIdx++;
					}
				}
				
				// Close file
				myFile.close();
				
				return result;
			}
			
			
			void write_csv(std::string filename, std::string colname, std::vector<int> vals) {
				// Make a CSV file with one column of integer values
				// filename - the name of the file
				// colname - the name of the one and only column
				// vals - an integer vector of values
				
				// Create an output filestream object
				std::ofstream myFile(filename);
				
				// Send the column name to the stream
				myFile << colname << "\n";
				
				// Send data to the stream
				for (int i = 0; i < vals.size(); ++i)
				{
					myFile << vals.at(i) << "\n";
				}
				
				// Close the file
				myFile.close();
			}
			
			void write_csv(std::string filename, std::vector<std::pair<std::string, std::vector<int>>> dataset) {
				// Make a CSV file with one or more columns of integer values
				// Each column of data is represented by the pair <column name, column data>
				//   as std::pair<std::string, std::vector<int>>
				// The dataset is represented as a vector of these columns
				// Note that all columns should be the same size
				
				// Create an output filestream object
				std::ofstream myFile(filename);
				
				// Send column names to the stream
				for (int j = 0; j < dataset.size(); ++j)
				{
					myFile << dataset.at(j).first;
					if (j != dataset.size() - 1) myFile << ","; // No comma at end of line
				}
				myFile << "\n";
				
				// Send data to the stream
				for (int i = 0; i < dataset.at(0).second.size(); ++i)
				{
					for (int j = 0; j < dataset.size(); ++j)
					{
						myFile << dataset.at(j).second.at(i);
						if (j != dataset.size() - 1) myFile << ","; // No comma at end of line
					}
					myFile << "\n";
				}
				
				// Close the file
				myFile.close();
			}
			
			int main() {
				
				// Make a vector of length 100 filled with 1s
				std::vector<int> vec(100, 1);
				
				// Write the vector to CSV
				write_csv("ones.csv", "Col1", vec);
				
				// Make three vectors, each of length 100 filled with 1s, 2s, and 3s
				std::vector<int> vec1(100, 1);
				std::vector<int> vec2(100, 2);
				std::vector<int> vec3(100, 3);
				
				// Wrap into a vector
				std::vector<std::pair<std::string, std::vector<int>>> vals = { {"One", vec1}, {"Two", vec2}, {"Three", vec3} };
				
				// Write the vector to CSV
				write_csv("three_cols.csv", vals);
				// Read three_cols.csv and ones.csv
				std::vector<std::pair<std::string, std::vector<int>>> three_cols = read_csv("three_cols.csv");
				std::vector<std::pair<std::string, std::vector<int>>> ones = read_csv("ones.csv");
				
				// Write to another file to check that this was successful
				write_csv("three_cols_copy.csv", three_cols);
				write_csv("ones_copy.csv", ones);
				
				return 0;
			}
			
			
		\end{lstlisting}
	\end{scriptsize}	
\end{document}