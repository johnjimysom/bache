\documentclass[a4paper]{article}

\usepackage[english]{babel}
\usepackage[utf8]{inputenc}
\usepackage{amsmath}

\usepackage{xcolor}

\definecolor{verde}{rgb}{0.25,0.5,0.35}
\definecolor{jpurple}{rgb}{0.5,0,0.35}
\definecolor{darkgreen}{rgb}{0.0, 0.2, 0.13}
%\definecolor{oldmauve}{rgb}{0.4, 0.19, 0.28}
\usepackage{listings}

\usepackage{tikz}
\def\checkmark{\tikz\fill[scale=0.4](0,.35) -- (.25,0) -- (1,.7) -- (.25,.15) -- cycle;} 
	

\newcommand{\cppTheme}{
	\lstset{
		language=Java,
		basicstyle=\ttfamily\small,
		keywordstyle=\color{jpurple}\bfseries,
		stringstyle=\color{red},
		commentstyle=\color{verde},
		morecomment=[s][\color{blue}]{/**}{*/},
		extendedchars=true,
		showspaces=false,
		showstringspaces=false,
		numbers=left,
		numberstyle=\tiny,
		breaklines=true,
		backgroundcolor=\color{cyan!10},
		breakautoindent=true,
		captionpos=b,
		xleftmargin=0pt,
		tabsize=2
}}

\newcommand{\test}{
	\lstset{ %
		language=R,                     % the language of the code
		basicstyle=\footnotesize,       % the size of the fonts that are used for the code
		numbers=left,                   % where to put the line-numbers
		numberstyle=\tiny\color{gray},  % the style that is used for the line-numbers
		stepnumber=1,                   % the step between two line-numbers. If it's 1, each line
		% will be numbered
		numbersep=5pt,                  % how far the line-numbers are from the code
		backgroundcolor=\color{white},  % choose the background color. You must add \usepackage{color}
		showspaces=false,               % show spaces adding particular underscores
		showstringspaces=false,         % underline spaces within strings
		showtabs=false,                 % show tabs within strings adding particular underscores
		frame=single,                   % adds a frame around the code
		rulecolor=\color{black},        % if not set, the frame-color may be changed on line-breaks within not-black text (e.g. commens (green here))
		tabsize=2,                      % sets default tabsize to 2 spaces
		captionpos=b,                   % sets the caption-position to bottom
		breaklines=true,                % sets automatic line breaking
		breakatwhitespace=false,        % sets if automatic breaks should only happen at whitespace
		title=\lstname,                 % show the filename of files included with \lstinputlisting;
		% also try caption instead of title
		keywordstyle=\color{blue},      % keyword style
		commentstyle=\color{darkgreen},   % comment style
		stringstyle=\color{red},      % string literal style
		escapeinside={\%*}{*)},         % if you want to add a comment within your code
		morekeywords={*,...}          % if you want to add more keywords to the set
}}

\title{Zebra TC-21 QR Scanner Set-Up}
\author{Johnjimy K. Som}

\begin{document}
	\maketitle
	
	\begin{scriptsize}%begin{abstract} (can be used here)
		Documentation to set-up a Zebra TC21 to scan QR Barcodes in hexadecimal, store in a .csv or .xlsx file, send the files into an SMB network, and parse in decimal to determine item codes from that scanned QR code. Note: \emph{Italicized Text} means it can be seen in the Zebra Device.
	\end{scriptsize}%\end{abstract}
	
	\section{Zebra TC-21 Configurations}
	
	\begin{scriptsize}
	Requirements to scan QR code:
	
	\begin{itemize}
		\item Zebra TC21 [Model:TC210K], Android Version 10
		\item Scanning Application
		\item File Explorer Application
		\item Server Message Block (SMB) configuration skills
	\end{itemize}
	
	\end{scriptsize}
	
	\subsection*{\emph{Data Wedge} Application}
		\begin{scriptsize}
		\emph{Data Wedge} is a built in application for Zebra devices, In order to have the scanning function to perform quickly and efficiently, the following changes needs to made:
		
		\begin{itemize}
			\item In the \emph{Data Wedge} application, Profile:Profile0 (default) should be selected
			\item Scroll down to \emph{Keystroke output} section, 'Inter character delay' is up to [10 ms]
			\item Scroll down and select \emph{Basic Data formatting} section  
			\item 'Send ENTER key' should be [\checkmark], not unchecked 
		\end{itemize}
		
	\end{scriptsize}

	\section{'Orca Scan' Android Package (APK)}
	\begin{scriptsize}
		\emph{Orca Scan} is an application created by Cambridge App Lab Limited. \emph{Orca Scan} will make the TC21 device: scan, export .csv, .xlsx to a file manager application that connects to an Server Message Block (SMB) service.
		
		\begin{itemize}
			\item Download \emph{Orca Scan v5.7.8} APK from a web browser
			\item Be sure to accept permissions on third-party downloads
			\item \emph{CAUTION:} downloading APK that is not from 'Google Play Store' is dangerous
			\item Install the 'Orca Scan v5.7.8.apk' file
		\end{itemize}
		\subsection*{\emph{Orca Scan} Versions}
			Orca Scan App 6.2.0 Update
			\\  2021-06-05
			\\- Added support for Honeywell barcode scanners
			\\- Fixed bug extracting weight from GS1 barcodes (thanks Jonas)
			\\- Restored `Scan into field` button when manually entering a barcode
			\\- Remove Orca yellow from light theme
			
	\end{scriptsize}
%Ignore sample template
%	\begin{scriptsize}
%		\test
%		\begin{lstlisting}[caption={Coding in R}, label=lst:rcode]
%			name = input('who are you? ')
%			hero_name= input('who is your favorite heroic servant in FGO? ')
%			print('hello ' + name + ' your favorite heroic servant is ' + hero_name)
%			
%			birth_year = input('birth year: ')
%			print(type(birth_year))
%			int(birth_year)
%			age = 2021 - int(birth_year)
%			print(type(age))
%			print(age)
%			
%		\end{lstlisting}
%	\end{scriptsize}

	\section{Parsing QR Bar-code}
	\begin{scriptsize}
	\cppTheme
	\begin{lstlisting}[caption={Snippet of the C++ to decode the bar-code}, label=lst:rcode]
	#include <iostream>
	#include <string.h>
	...
	string GetBinaryStringFromHexString(string);
	
	int main(int argc, char *argv[]) // input is a 32 character string

	{
		unsigned int encodeVer, printArea, itemCode, packingDiv, productionYear, quantity, serialNumber, checkSum;
		string inHex, inBinary;
		
		for (int i = 1; i < argc; ++i)
		{
			inHex = argv[i];
			inBinary = GetBinaryStringFromHexString(inHex);
			//cout << inHex << "\n";	
			...
		} 
		return 0;
	}
	
	string GetBinaryStringFromHexString(string sHex)
	{
		string sReturn = "";
		for (unsigned int i = 0; i < sHex.length(); ++i)
		{
			switch (toupper(sHex[i]))
			{
				case '0': sReturn.append("0000"); break;
				case '1': sReturn.append("0001"); break;
				case '2': sReturn.append("0010"); break;
				case '3': sReturn.append("0011"); break;
				case '4': sReturn.append("0100"); break;
				case '5': sReturn.append("0101"); break;
				case '6': sReturn.append("0110"); break;
				case '7': sReturn.append("0111"); break;
				case '8': sReturn.append("1000"); break;
				case '9': sReturn.append("1001"); break;
				case 'A': sReturn.append("1010"); break;
				case 'B': sReturn.append("1011"); break;
				case 'C': sReturn.append("1100"); break;
				case 'D': sReturn.append("1101"); break;
				case 'E': sReturn.append("1110"); break;
				case 'F': sReturn.append("1111"); break;
			}
		}
		return sReturn;
	}
		
	\end{lstlisting}
\end{scriptsize}

\subsection*{I. Decode Bar-Code: \emph{Visual Studio}}
\begin{scriptsize}
	"\emph{Visual Studio} dev tools \& services make app development easy for any platform \& language."
	
	\begin{itemize}
		\item Integrated Development Environment (IDE)
		\item Scroll down to \emph{Keystroke output} section, 'Inter character delay' is up to [10 ms]
		\item Scroll down and select \emph{Basic Data formatting} section  
		\item 'Send ENTER key' should be [\checkmark], not unchecked 
	\end{itemize}
	
\end{scriptsize}
\subsection*{II. Decode Bar-Code: Future Updates}
\begin{scriptsize}
	Future updates regarding to the parser is a built in application for Zebra devices, In order to have the scanning function to perform quickly and efficiently, the following changes needs to made:
 	
\end{scriptsize}

\section{Server Message Block (SMB) Configuration}
\begin{scriptsize}
	"In computer networking, Server Message Block \emph{(SMB)}, one version of which was also known as Common Internet File System (CIFS), is a communication protocol for providing shared access to files, printers, and serial ports between nodes on a network."
	

	\subsection*{Using \emph{FE File Explorer} for SMB}
	'FE File Explorer' can be installed via 'Google Play Store'
	\\  
	\\- 
	\\- 
	\\- ... [INCOMPLETE]
	\\- 
	
\end{scriptsize}

\end{document}